\documentclass{article}
\usepackage{float}
\usepackage{fancyhdr}
\usepackage{geometry}
\usepackage{amsmath}
\usepackage{amssymb}
\usepackage{bbm}
\usepackage{listings}
\usepackage{xcolor}
\usepackage{tikz}
\usepackage[parfill]{parskip}

\parskip = \baselineskip

\pagestyle{fancy}
\lhead{IE 420 HW5  \ \ \ \ \ Name: Karl Oskar Julius Olson}
\rhead{Score: \ \ \ \ \ \ \ \ \ \ \ \ \ }
\renewcommand{\headrulewidth}{0.4pt}
\renewcommand{\footrulewidth}{0.4pt}

\begin{document}

\thispagestyle{fancy}

\section{Binomial I}

$$S_0 = \$50, \ \delta=1/2, \ u=1.2, \ d=0.8, \ r=0.01, \ K=\$50$$
$$p^* = \frac{e^{r \delta} - d}{u - d} = 0.512531$$
Backward induction is used:
$$f_{uu} = (K - u^2S_0)^+ = 0, \ f_{dd} = (K-d^2S_0)^+ = 18, \ f_{ud} = (K - udS_0)^+ = 2$$
\begin{align*}
	f_u &= e^{-r\delta}\left(p^* \cdot f_{uu} + (1-p^*) \cdot f_{ud}\right) = 0.970075\\
	f_d &= e^{-r\delta}\left(p^* \cdot f_{ud} + (1-p^*) \cdot f_{dd}\right) = 9.750624\\
	f_0 &= e^{-r\delta}\left(p^* \cdot f_{u} + (1-p^*) \cdot f_{d}\right) = 5.224132\\
\end{align*}
The value of the option is thus: $f_0 = \$5.224132$

The initial delta hedge is constructed according to:
$$\Delta_0 = 100 \cdot \frac{f_u-f_d}{S_0 \cdot (u-d)} = -43.902745422390126$$
After the stock price goes down this is adjusted:
$$\Delta_d = 100 \cdot \frac{f_{ud}-f_{dd}}{dS_0 \cdot (u-d)} = -100$$
This amounts to initially short selling 43.903 shares and then short selling additional shares so that the total shorted shares equals 100.

\section{Lookback Binomial}
$$S_0 = \$100, \ \delta=3/12, \ u=1.1, \ d=0.9, \ r=0.01, \ K=\$95$$
\begin{align*}
	f_{uu} &= (max(S_0, uS_0, u^2S_0) - K)^+ = 26\\
	f_{dd} &= (max(S_0, dS_0, d^2S_0) - K)^+ = 5\\
	f_{ud} &= (max(S_0, uS_0, udS_0) - K)^+ = 15\\
	f_{du} &= (max(S_0, dS_0, duS_0) - K)^+ = 5\\
\end{align*}
First, the risk neutral probability is calculated:
$$p^* = \frac{e^{r\delta}-d}{u-d} = 0.512516$$

$$f_0 = e^{-2r\delta}\mathbb{E}^*\left[f(S_{2\delta)}\right] = e^{-2r\delta}\left({p^*}^2f_{uu} + p^*(1-p^*)(f_{ud} + f_{du}) + (1-p^*)^2f_{dd}\right) = \$12.949641$$

The delta hedge is calculated as:
$$\Delta_0 = 100 \cdot \frac{f(max(S_0, uS_0))-f(S_0, dS_0)}{S_0 \cdot (u-d)} = 50$$
Which equates to buying (long) 50 shares.

\section{Binomial Stock Index}
$$S_0 = \$2500, \ \delta=3/12, \ u=1.15, \ d=0.9, \ r=0.01, \ q=0.03, \ K=\$2500$$

First the risk neutral probability is calculated:
$$p^* = \frac{e^{(r-q) \delta} - d}{u - d} = 0.380050$$
The payoff at maturity is then calculated as:
\begin{align*}
	f_{uu} &= (u^2S_0 - K)^+ = \$806.25 \\
	f_{dd} &= (d^2S_0 - K)^+ = \$0 \\
	f_{ud} &= (u^2S_0 - K)^+ = \$87.5\\
\end{align*}
We can now continue backwards one step and calculate payoff at $t=\delta$. As this regards an American option the choice of exercising early has to be taken into consideration.
\begin{align*}
	f_u &= \max\left( (uS_0 -K)^+, e^{-r\delta}(p^*f_{uu} + (1-p_{star})f_{ud}) \right) = \max(375, 359.760325) = \$375 \\
	f_d &= \max\left( (dS_0 -K)^+, e^{-r\delta}(p^*f_{ud} + (1-p_{star})f_{dd}) \right) = \max(0, 33.171336) = \$33.171336\\
\end{align*}
Finally the value at time $t=0$ is calculated:
$$f_0 = \max\left( (S_0 -K)^+, e^{-r\delta}(p^*f_{u} + (1-p_{star})f_{d}) \right) = \max(0, 162.676092) = \$162.676092$$
The value of the american option is therefore: $C = \$162.676092$. At time $t=\delta$ the option could be exercised if the price of the index has gone up. The same does \textbf{not} hold when the price has gone down at $t=\delta$. 

The price for the equivalent European option is simpy calculated as:
$$c = e^{-2r\delta}\mathbb{E}^*\left[f(S_{2\delta})\right] = e^{-2r\delta}\left({p^*}^2f_{uu} + 2p^*(1-p^*)f_{ud} + (1-p^*)^2f_{dd}\right) = 156.898726$$

The price payed for the right to exercise early is: $C-c = 5.777366$

\section{CRR \& Black-Scholes}
$$S_0 = \$1.25, \ \delta=2/12, \ \sigma=0.15, \ r=0.01, \ q=0.015, \ K=\$1.25, \ N = 3$$
$$u = e^{\sigma\sqrt{\delta}} = 1.063151, \ d = e^{-\sigma\sqrt{\delta}} = 0.940600$$
$$p^* = \frac{e^{(r-q)\delta}-d}{u-d} = 0.477898$$
After calculating $p^*, u, d$, we can now utilize backward induction to get $f_0$:

$$f_N = \left[ (K-u^3S_0)^+, (K-u^2dS_0)^+, (K-ud^2S_0)^+, (K-d^3S_0)^+ \right]$$
$$f_0 = e^{-r(N\delta)}\sum_{j=0}^N \binom{N}{j}{p^*}^j(1-p^*)^{N-j}f_{N,j} = \$0.058580$$

Black sholes:
$$d_1 = \frac{\ln(S_0/K) + (r-q+\frac{1}{2}N\delta)}{\sigma\sqrt{N\delta}} = 0.029463, \ d_2 = d_1-\sigma\sqrt{N\delta} = -0.076603$$
$$p = -S_0e^{-qN\delta}\mathcal{N}_{cdf}(-d_1) + Ke^{-rN\delta}\mathcal{N}_{cfd}(-d_2) = \$0.054106$$

The difference between CRR and Black-Sholes is therefore = $\$0.004474$

\section{Portfolio}

$$\bar{r}_A = 0.04, \ \bar{r}_B = 0.02, \ \bar{r}_C = 0.01, \ \ \vec{r} = \left[0.04, 0.02, 0.01\right]$$
$$\sigma_A = 0.2, \ \sigma_B = 0.12, \sigma_C = 0.09, \ \ \vec{\sigma} = \left[0.2, 0.12, 0.09\right]$$
$$\rho_{A, B} = 0.3, \ \rho_{A, C} = 0.1, \ \rho_{B, C} = 0.2$$
$$\rho = \begin{bmatrix} 1 & 0.3 & 0.1 \\ 0.3 & 1 & 0.2 \\ 0.1 & 0.2 & 1 \end{bmatrix}$$

The covariance matrix $\Sigma$ is calculated as (using linear algebra): $\vec{\sigma}^T \vec{\sigma} \odot \mathbf{\rho}$. $\rho$ in this case refers to the correaltion-matrix. $\odot$ denotes elementwise multiplication

For the first portfolio the weights are: $\vec{w_{p1}} = \left[0.25, 0.25, 0.5\right]$.
Using linear algebra, the mean and variance are calculated as:
$$\bar{r}_{p1} = \vec{w_{p1}} \cdot \vec{r} = 0.02, \ \sigma^2_{p1} = \vec{w_{p1}}^T\vec{w_{p1}} \odot \Sigma = 0.007315$$

The second portfolio is weighted as: $\vec{w_{p2}} = \left[1/6, 1/2, 1/3 \right]$.
Again the mean and variance are calculated as:
$$\bar{r}_{p2} = \vec{w_{p2}} \cdot \vec{r} = 0.02, \ \sigma^2_{p2} = \vec{w_{p2}}^T\vec{w_{p2}} \odot \Sigma = 0.0077311$$

The standard deviations for the two portfolios are calculated as: $\sigma_{p1} = 0.085528, \ \sigma_{p2} = 0.087927$

As the expected return of both portfolios is the same, the best choice would be the one that exhibits the lowest variability and therefore lowest standard deviation. Therefore, portfolio one is the best choice of the two. 

\end{document}


