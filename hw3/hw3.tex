\documentclass{article}
\usepackage{float}
\usepackage{fancyhdr}
\usepackage{geometry}
\usepackage{amsmath}
\usepackage{amssymb}
\usepackage{bbm}
\usepackage{listings}
\usepackage{xcolor}
\usepackage{tikz}
\usepackage[parfill]{parskip}

\parskip = \baselineskip

\pagestyle{fancy}
\lhead{IE 420 HW3 \ \ \ \ \ Name: Karl Oskar Julius Olson}
\rhead{Score: \ \ \ \ \ \ \ \ \ \ \ \ \ }
\renewcommand{\headrulewidth}{0.4pt}
\renewcommand{\footrulewidth}{0.4pt}

\begin{document}

\thispagestyle{fancy}


\section{Forward Price}

$$S_0 = \$180, r = 1.6\%, c = \$0.51, T = 6$$
The forward price is given by, $F_0 = (S_0 - D_0)e^{rT}$
$$D_0 = c*e^{-r * 1/12} + c*e^{-r * 4/12} \approx 1.0166 $$
$$F_0 = (180 - 1.0166)e^{0.016 * 6/12} \approx 180.4210022$$

After 2 month: $S_t = \$160, t=2, K=F_0$
$$D_t = c * e^{-r * 2/12} = 0.5086$$
$$F_t = e^{r(T-t)}(S_t - D_t) = e^{0.016 * 4/12}(160-0.5086) = 158.643002$$
The value at time t, is given by,
$$V_t = e^{-r(T-t)}(F_t - K) = e^{-0.016 * 4/12}(158.643002-180.4210022) = -21.662160$$

\section{GBP Forward I}

$$S_0 = 1.3, T=6, r = 1.6\%, q = 1.0\%$$

For zero-value, forward price should be: $F_0 = S_0*e^{(r-q)T} = 1.3 * e^{(0.016-0.01)/2} \approx \$1.30391 $. As the qouted price is lower than this, an arbitrage opportunity exists for a long forward contract.

At time = 0: 
\begin{itemize}
	\item Long forward to buy 1 M GBP in 6 months
	\item Loan $e^{-qT}$ M GBP and sell to receive $S_0*e^{-qT}$ M USD.
	\item Deposit the dollars for 6 months with interest rate $r$.
\end{itemize}

At time = T: 
\begin{itemize}
	\item Withdraw $S_0 * e^{(r-q)T}$ M USD.
	\item Buy 1 M GBP for forward price $\$1.30151$ million and pay back loan
\end{itemize}
At, the end this amounts to the amount withdrawn from the account subtracted by the cost to pay back the loan:
$$ S_0 * e^{(r-q)T} MUSD - 1.30151 MUSD \approx \$2395.855854$$


\section{LIBOR FRA}
$$r_1 = 1.6\%, \ r_2 = 2\%, \ t_1 = 6, \ t_2 = 18, \ L = \$1,000000$$

The interest rate to ensure a zero-cost FRA is given by:
$$r_K = \frac{r_2t_2 - r_1t_1}{t_2-t_1} = \frac{0.02 * 18/12 - 0.016*6/12}{(18-6)/12} = 2.2\%$$

In 6 months: $r_M = 2.4\%, \ r=2.0\%$

The value for the borrower at time $t_2$ is calculated as: $ V(t_2) = L\left( e^{r_M(t_2-t_1)} - e^{r_K(t_2-t_1)} \right)$. To get the value of the FRA for the borrower at the time of settlement ($t_1$), this has to be discounted by the risk free interest rate:
$$V(t_1) = e^{-r(t_2-t_1)}V(t_2) = Le^{-r(t_2-t_1)}\left( e^{r_M(t_2-t_1)} - e^{r_K(t_2-t_1)} \right) = $$
$$ = 1000000 * e^{-0.02} \left( e^{0.024} - e^{0.022} \right) \approx \$2006.009343$$

\section{GBP Forward II}
$$S_0 = \$1.3/GPB, \ r_b = \%1.7, \ r_d = \%1.5, \ y=\%1$$
% If the forward price is lower than zero-cost a opportunity to long the contract exists. This is carried out by borrowing $e^{-qT}$ M GBP at $\%1$, exchanging them for $S_0e^{-qT}$ M USD and then depositing them with rate $r_deposit$. At maturity, the loan is paid back for 
There are two arbitrage opportunities: If the price is low enough the arbitrager can long the forward and if the price is too high the arbitrager can short the forward.
\begin{itemize}
	\item Profit - long arbitrage: $S_0e^{r_{d}-q}T - F_0= 1.3e^{0.005/12} - F_0$
	\item Profit - short arbitrage: $F_0 - S_0e^{r_{b}-q}T = F_0 - 1.3e^{0.007/12}$
\end{itemize}

To elimiate both opportunities, the following must hold: 
$$\begin{cases} F_0 > 1.3e^{0.005/12} \\ F_0 < 1.3e^{0.007/12} \end{cases}$$
This gives the answer as: $1.300542 \leq F_0 \geq 1.300759$


\section{Put Arbitrage}
$$K_1 = \$315, \ P_1 = \$5.60 \ K_2 = \$317.5, \ P_2 = \$5.40$$
An arbitrage strategy for this situation would be to short the put option with the lower strike price, and buy the put option with the higher strike price. 

\textit{At time 0}: Short 315-puts and receive $\$5.60 / option$. Buy equally many 317.5-puts for $\$5.40/option$.
The payoff for a short put: $-(K-S)^+$, and for a long put: $(K-S)^+$. For this setup the combined P\&L thus amounts to:
$$(K_2-S)^+ -(K_1-S)^+ + (P_1 - P_2) = (317.5-S)^+ - (315-S)^+ + 0.2$$
Considering the stock price $S$ when exercising the option, there are three possibilites.
\begin{enumerate}
	\item $S > \$317.5$. This means that none of the options are exercised and the P\&L becomes: \\ $P\&L = \$0.2$
	\item $\$315 < S < \$317.5$. In this situation, only the 317.5-put is exercised: \\ $P\&L = 317.5 - S + 0.2 > \$0.2$
	\item $S < \$315$. Finally, this case means that both options are exercised: \\$P\$L = 317.5 - S - 315 + S + 0.2 = \$2.7$
\end{enumerate}
Thus, no matter the share price $S$ the strategy implies a profit of at least $\$0.2$

\end{document}


