\documentclass{article}
\usepackage{float}
\usepackage{fancyhdr}
\usepackage{geometry}
\usepackage{amsmath}
\usepackage{amssymb}
\usepackage{bbm}
\usepackage{listings}
\usepackage{xcolor}
\usepackage{tikz}
\usepackage[parfill]{parskip}

\parskip = \baselineskip

\pagestyle{fancy}
\lhead{IE 420 HW4 \ \ \ \ \ Name: Karl Oskar Julius Olson}
\rhead{Score: \ \ \ \ \ \ \ \ \ \ \ \ \ }
\renewcommand{\headrulewidth}{0.4pt}
\renewcommand{\footrulewidth}{0.4pt}

\begin{document}

\thispagestyle{fancy}

\section{Put Pricing}

$$S_0 = \$90, \ c = \$5.2, \ K = S_0, \ r = 0.9\%, \ T = 2/12 $$
$$D_0 = c_1 * e^{-r * t_1} = 1.1 * e^{-0.009 * 1/12} \approx \$1.099175 $$

The price is determined using the put-call parity, for discrete income assets: 
$$ c + e^{-rT}K = p + S_0 - D_0 \implies p = c + e^{-rT}K - (S_0 - D_0)$$
$$p = 5.2 + 90 * e^{-0.009 * 2/12} - (90 - 1.099175) \approx \$6.164277$$

\section{GBP Arbitrage}

$$S_0 = \$1.3050, \ r = 0.8\%, \ q = 1.2\%, \ K = 1.3, \ T = 1, \ c = \$0.048, \ p=\$0.03$$
The put-call parity for assets with continous yield: 
$$c + e^{-rT}K  = p + S_0 e^{-qT}$$
$$0.048 + 1.3 * e^{-0.008} = \$1.337641 > 0.03 + 1.305*e^{-0.012} = \$1.319434$$
As the value of the portfolio with the call option is higher, an arbitrage strategy is to sell the call and buy the put and the asset (GBP).
\begin{itemize}
	\item \textbf{At time T=0} - Sell call option, buy put and asset. This is done by borrowing $p + S_0 -c = \$1.287$.
	\item \textbf{At time T=1 year}: Value of loan: $1287 * e^{rT} \approx \$1.297337$\begin{itemize}
		\item If $S_T < 1.3$: Sell GBP and get $\$1.3 * e^{qT}$ and pay off loan. \\Profit: $1.3*e^{qT} - 1.297337 \approx \$0.018357$ 
		\item If $S_T > 1.3$: Sell GBP and get $\$S_T * e^{qT}$ and pay off loan. \\Profit: $S_T*e^{qT} - 1.297337 \geq \$0.018357$ 
	\end{itemize}
\end{itemize}
A profit of at least $\$0.018357$/unit is guaranteed, and this is thus an arbitrage opportunity.

\section{Call Bear Spread}
$$ S_0 = \$703, \ K_2 = \$700, \ K_1=\$680, \ p_2 = \$38, \ p_1 = \$52$$
A bear spread using calls is constructed by buying the call with the higher strike price, and selling the one with the lower. This results in a "credit" of $p_1 - p_2 = 52 - 28 = \$14$.

The P\&L for this setup is as follows: $(S_T - K_2)^+ - (S_T - K_1)^+ + 14$.

There are three cases to consider:
$$P\&L = \begin{cases} S_T - 700 - S_T + 680 + 14 = -6, \ \ \ \ \ \ \ if \ S_T > \$700 \\ -S_T - 680 + 14 = 694 - S_T, \ \ \ \ \ \ \ \ \ \ \  \ if \ \$680<S_T<\$700 \\ 0 - 0 + 14 = 14, \ \ \ \ \ \ \ \ \ \ \ \ \ \ \ \ \ \ \ \ \ \ \ \ \ \ \ \  if \ S_T < \$680\end{cases}$$
From this, it is easy to identify the maximum profit as $\$14/option$, the maximum loss as $-\$6/option$ and the break even point as $S_T=\$694$

\section{Binomial I}
$$ S_0 = \$20, \ u = 1.2, \ d = 0.8, \ r=1\%, \ Payoff = max(0, 400 - S^2)$$

$$f(S_{\delta}): f_u = f(uS_0) = max(0, 400 - (1.2*20)^2) = 0.0$$
$$f_d = f(dS_0) = max(0, 400 - (0.8*20)^2) = \$144$$
To replicate the option, we construct a portfolio with $\Delta$ number of shares, and borrow $\Psi$ dollars. $\Delta$ and $\Psi$ are given by:
\begin{align*}
	\Delta &= \frac{f_u - f_d}{S_0 (u -d)} = \frac{0 - 144}{20 * (1.2 - 0.8)} = -18 \\
	\Psi &= \frac{df_u - uf_d}{e^{r\delta}(u-d)} = \frac{0 - 1.2 * 144}{e^{0.01 * 6/12} * (1.2-0.8)} \approx -\$429.845391
\end{align*}
The value of the derivative (price of the power put) at time zero for no arbitrage should be:
$$f_0 = S_0\Delta - \Psi = 20*-18 - (-429.845391) \approx \$69.845391 $$

\section{Binomial II}
$$u = 1.1, \ d = 0.8, \ T=\delta=3/12, \ r = 1\%, \ K = \$95$$
$$f_u = max(K - uS_0, 0) = 0, \ f_d = max(K - u * dS_0, 0) = 15$$
For the risk neutral pricing, we first need the risk neutral probability $p^*$.
$$p^* = \frac{e^{r\delta} - d}{u - d} = \frac{e^{0.01 * 3/12}-0.8}{1.1-0.8} \approx 0.675010$$
The risk neutral price is then given as the risk neutral expectation of the payoff discounted at the risk free rate:
$$f_0 = e^{-r\delta}\mathbb{E}[f(S_{\delta})] = e^{-r\delta}\left(p^*f_u + (1-p^*)f_d\right) = e^{-0.01*3/12}\left((1-0.675010)*15\right) = \$4.862672$$


\end{document}


